\documentclass[12pt,answers]{exam}
%\documentclass[12pt]{exam}
\usepackage{amsmath}
\usepackage{amssymb}
\usepackage{listings}
\usepackage{graphicx}
\usepackage{wrapfig}
\usepackage{xypic}

\lstdefinestyle{basic}{showstringspaces=false,columns=fullflexible,language=Java,
%escapechar=@,xleftmargin=1pc,%
%basicstyle=\small\bfseries\itshape,%
%keywordstyle=\underbar,
mathescape=true,%
lineskip=-1pt,%
keepspaces=true,%
%basicstyle=\small\sffamily,%
basicstyle=\footnotesize\ttfamily,%
commentstyle=\mdseries\rmfamily,%
%morekeywords={class,object,with,fun,new,end,int,bool,string},%
deletekeywords={error,apply,values,and,equal,eq},%
moredelim=**[is][\color{red}]{`}{`}
}
\lstset{style=basic}

\firstpageheader{\bf\large Data Structures\\ CSCI H343, Fall 2023 \\[1ex]}%
{}%
{\bf Homework 2 \\[1ex]Name:\enspace\hbox to 2in{\hrulefill}}
\firstpageheadrule

\runningheader{H343 Data Structures\\[1ex]}%
{Homework 2 \\ }%
{Fall 2023 \\[1ex] Name:\enspace\hbox to 2in{\hrulefill}}

\runningfooter{}{Page \thepage\ of \numpages}{}
\runningheadrule
\runningfootrule

\extraheadheight{.75in}
\extrafootheight{-.5in}

\addpoints
\boxedpoints

\begin{document}

\vspace{20pt}

\begin{center} 
This homework has \numquestions\ questions, for a total of \numpoints\ 
points. 
\end{center} 


% 10 questions
% ~~ 2 questions write code
% topics:
%
% projects
%   * floodit                        
%   * segment intersection (BST, AVL)
%
% BST                              
% AVL                               
% big-O
%    algorithm -> big-O             
%    prove function in big-O        
% Java basics:
%   * arrays  
%   * lists
%   * dictionaries, sets
%   * loops
%   * recursion                       
%   * aliasing, mutation,             
%     parameter passing,              
%     identity vs. equality           
%   * classes & objects & self     
% arrays
%   * rotation                         
% linked lists
%   singly linked lists              
%   iterators                         
% proof of correctness, loop invariants

% 2023
% 1. implement product algorithm on Iter
% 2. remove from AVL tree
% 3. big-O proof
% 4. Java basics: recursion
% 5. Testing array search
% 6. big-O of O(n^2) flood algorithm
% 7. Next/previous in BST
% 8. linked lists, filter, fill in the blanks
% 9. big-O of binary search
% 10. big-O of a simple function that is O(2^n)
% 11. proof of correctness of linked-list insert using induction

% 2022
% 1. java parameter passing, builtins and objects
% 2. implement iterator on array, fill in blanks
% 3. BST find, implement
% 4. big-O proof: if f <~ g, then f + g <~ g
% 5. big-O of code: double nested loop that is Θ(n log n)
% 6. linked lists, insert sorted, fill in blanks
% 7. AVL insert and rotation
% 8. LinkedList method time complexity
% 9. ArrayList method time complexity
% 10. binary search, fill in blanks
% 11. loop invariant
  
% 2021
% 1. java objects and fields
% 2. next prev
% 3. fill in blanks, next, first, nextAncestor
% 4. loop invariant
% 5. big-O of flood
% 6. find_first_equal on Iterator interface  
% 7. recognize BST, AVL tree
% 8. big O proof  
% 9. AVL remove node, rotate, etc.
% 10. big-O of anagram detection
  
  
% 2020:
% 1. identity versus equality, aliasing
% 2. big-O collections
% 3. rotate array, fill in the blanks
% 4. max_heapify, fill in the blanks
% 5. implement the \texttt{find} method for BST
% 6. implement the \texttt{put} method for hash tables
% 7. identify BST's, max heap's
% 8. draw insertion into an AVL tree
% 9. big-O of anagram detector, nested loops O(n^2)
% 10. big-O prove 4n+2 in O(n^2)


\begin{questions}

  \question[6] Prove that the following \texttt{insert} function
produces a list whose length is one greater than the length of the
input \texttt{list}. In other words, prove that
\begin{lstlisting}
  length(insert(x, list)) == 1 + length(list)
\end{lstlisting}

\noindent Do the proof by induction on the \texttt{length} of the
\texttt{list}. Indicate where you use the induction hypothesis.

\begin{lstlisting}
public static Node<Integer> insert(int x, Node<Integer> list) {
    if (list == null) {
        return new Node<Integer>(x, null);
    } else if (x <= list.data) {
        return new Node<Integer>(x, list);
    } else {
        return new Node<Integer>(list.data, insert(x, list.next));
    }
}
\end{lstlisting}
The \texttt{length} of a list is defined as follows
\begin{lstlisting}
public static <T> int length(Node<T> node) {
    if (node == null) {
        return 0;
    } else {
        return 1 + length(node.next);
    }
}
\end{lstlisting}

\begin{solution}
  \scriptsize
  
\textbf{Base Case:} \textbf{(2 points)}
We have \texttt{length(list) == 0} and therefore \texttt{list == null}.
\begin{lstlisting}
length(insert(x, null)) == length(new Node(x, null)) == 1 + length(null)
\end{lstlisting}

\textbf{Induction Step:}
Suppose we have a list \texttt{M} of length $k + 1$.
From that we know two things: \texttt{M != null}
and the length of \texttt{M.next} is $k$.
We have two cases to consider regarding whether \texttt{x <= M.data}.

Suppose \texttt{x <= M.data}. \textbf{(2 points)}
\begin{lstlisting}
length(insert(x, M)) == length(new Node(x, M)) == 1 + length(M)
\end{lstlisting}


Next, suppose that \texttt{x > M.data}.  \textbf{(2 points)}
\begin{lstlisting}
length(insert(x, M))
== length(new Node(M.data, insert(x, M.next)))
== 1 + length(insert(x, M.next))
== 1 + (1 + length(M.next))           // by the induction hypothesis
== 1 + length(M)
\end{lstlisting}

  
\end{solution}
\end{questions}

\end{document}
